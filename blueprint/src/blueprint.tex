\documentclass{article}

\input{../../.lake/build/dressed/library/SBSTest.tex}

\usepackage{amsmath, amsthm, amssymb}
\usepackage{hyperref}

\theoremstyle{definition}
\newtheorem{definition}{Definition}[section]
\newtheorem{theorem}{Theorem}[section]
\newtheorem{lemma}{Lemma}[section]

\title{SBS-Test: Status Demo}
\author{Side-By-Side Blueprint Project}

\newcommand{\N}{\mathbb{N}}

\begin{document}
\maketitle

\chapter{Introduction}

This minimal test project demonstrates all features of the Side-by-Side Blueprint:
\begin{itemize}
\item All 8 node status types in the dependency graph
\item Disconnected component detection (the cycle is isolated)
\item Cycle detection (cycleA and cycleB form a mutual dependency)
\item Dashboard features (key theorems, messages, blocked, issues, debt, misc)
\end{itemize}

\chapter{Main Graph: Status Types}

This chapter demonstrates the 8 node status types with a tree-structured dependency graph.

\section{Base Definition}

The core definition that all theorems depend on.

\inputleannode{def:base}

\section{First-Level Dependencies}

Three theorems branching from the base definition, each with different status flags.

\inputleannode{thm:notready}

\inputleannode{thm:stated}

\inputleannode{thm:ready}

\section{Second Level: Sorry Status}

\inputleannode{thm:sorry}

\section{Convergence Point}

\inputleannode{thm:proven}

\section{Downstream Statuses}

\inputleannode{thm:fullyproven}

\inputleannode{thm:mathlibready}

\inputleannode{thm:inmathlib}

\chapter{Validation Test Cases}

These nodes test the graph validation features.

\section{Disconnected Cycle}

These two nodes form a cycle AND are disconnected from the main graph.
This tests both cycle detection and disconnected component detection.

\inputleannode{thm:cycleA}

\inputleannode{thm:cycleB}

\end{document}
