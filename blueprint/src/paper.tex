\documentclass{book}

\usepackage{amsmath, amsthm, amssymb}
\usepackage{hyperref}

\theoremstyle{definition}
\newtheorem{definition}{Definition}[section]
\newtheorem{theorem}{Theorem}[section]
\newtheorem{lemma}{Lemma}[section]
\newtheorem{corollary}{Corollary}[section]

\title{SBS-Test: ar5iv Paper Demo}
\author{Side-By-Side Blueprint Project}

\newcommand{\Z}{\mathbb{Z}}
\newcommand{\N}{\mathbb{N}}

\begin{document}

\chapter{Introduction}

This paper demonstrates the ar5iv-style paper generation feature.
Theorems are rendered with verification badges and links back to the blueprint.

\chapter{Basic Definitions}

We begin with the core definition that underlies all results.

\section{The Small Predicate}

\paperstatement{def:base}

\chapter{Status Examples}

This chapter demonstrates different proof statuses.

\section{Work In Progress}

The following theorem has an incomplete proof:

\paperfull{thm:sorry}

\section{Complete Proofs}

These theorems have been fully formalized:

\paperfull{thm:proven}

\paperfull{thm:fullyproven}

\chapter{Mathlib Integration}

Results ready for or already in Mathlib.

\paperstatement{thm:mathlibready}

\paperstatement{thm:inmathlib}

\chapter{Validation Cases}

The following nodes test cycle detection:

\paperfull{thm:cycleA}

\paperfull{thm:cycleB}

\end{document}
