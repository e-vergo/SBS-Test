\documentclass{article}

\input{../../.lake/build/dressed/library/SBSTest.tex}

\usepackage{amsmath, amsthm, amssymb}
\usepackage{hyperref}

\theoremstyle{definition}
\newtheorem{definition}{Definition}[section]
\newtheorem{axiom}{Axiom}[section]
\newtheorem{theorem}{Theorem}[section]
\newtheorem{lemma}{Lemma}[section]
\newtheorem{corollary}{Corollary}[section]

\title{SBS-Test Blueprint}
\author{Side-By-Side Blueprint Project}

\newcommand{\N}{\mathbb{N}}

\begin{document}
\maketitle

\chapter{Introduction}

This minimal test project demonstrates the Side-by-Side Blueprint system with
a structured dependency graph containing 22 nodes across three branches.
The project exercises all 6 status colors, diamond merge patterns,
cross-branch dependencies, and the Lean \texttt{axiom} keyword.

\chapter{Foundations}

This chapter establishes the base axiom and the foundation lemma
that all subsequent results depend on.

\section{Base Axiom}

A base axiom with no Lean code (notReady status by default).
This tests the TeX-only node feature.

\begin{axiom}[base\_axiom]
\label{base_axiom}
Every formalization project begins with informal axioms that
are not yet expressed in Lean. This base axiom represents
such an unformalized starting point.
\end{axiom}

\section{Foundation}

The foundation lemma connects the base axiom to the rest
of the dependency graph. It is manually marked as notReady.

\inputleannode{foundation}

\chapter{Branch A: Arithmetic}

This branch demonstrates sorry propagation through a dependency chain.
Even when downstream nodes have complete proofs, the sorry in an
ancestor prevents fullyProven status.

\section{Natural Number Identity}

\inputleannode{nat_identity}

\section{Addition with Zero}

\inputleannode{add_zero}

\section{Commutativity of Addition}

\inputleannode{add_comm}

\section{Associativity of Addition}

\inputleannode{add_assoc}

\chapter{Branch B: Set Theory}

This branch tests the Lean \texttt{axiom} keyword and demonstrates
how axioms integrate into the dependency graph.

\section{Set Theory Basics}

\inputleannode{set_basics}

\section{Choice Axiom}

This section uses the Lean \texttt{axiom} keyword, which Dress
auto-detects via \texttt{ConstantInfo.axiomInfo}.

\inputleannode{choice_axiom}

\section{Element Self-Membership}

\inputleannode{elem_self}

\section{Subset Reflexivity}

\inputleannode{subset_refl}

\chapter{Branch C: Logic}

This branch has no sorry and no manual status overrides in its ancestry.
All nodes auto-compute to fullyProven via the graph analysis.

\section{Proven Leaf}

The starting point of the fully proven chain.

\inputleannode{proven_leaf}

\section{Transitivity of Implication}

\inputleannode{imp_trans}

\section{Weakening Principle}

\inputleannode{weakening}

\section{Contrapositive}

\inputleannode{contrapositive}

\section{Disjunction Introduction}

This node demonstrates a diamond pattern: both contrapositive and
weakening depend on imp\_trans, and this node depends on both.

\inputleannode{disjunction_intro}

\chapter{Merging and Main Results}

These results merge the three branches, creating diamond patterns
and cross-branch dependencies.

\section{Core Theorem}

Merges Branch A (add\_assoc) and Branch B (subset\_refl) in a diamond pattern.

\inputleannode{core_theorem}

\section{Synthesis}

Merges all three branches by depending on core\_theorem and weakening.

\inputleannode{synthesis}

\section{Advanced Composition}

\inputleannode{advanced_composition}

\section{Main Result}

\inputleannode{main_result}

\section{Mathlib-Ready Theorem}

This theorem depends only on the fully proven chain (Branch C).

\inputleannode{mathlib_ready}

\chapter{Module Reference Test}

This chapter tests the \texttt{\textbackslash inputleanmodule\{ModuleName\}} command,
which imports all blueprinted declarations from a Lean module at once.

\section{All Declarations from Module}

The following declarations are imported via \texttt{\textbackslash inputleanmodule\{SBSTest.ModuleRefTest\}}:

\inputleanmodule{SBSTest.ModuleRefTest}

\end{document}
